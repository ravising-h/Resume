%-------------------------------------------------------------------------------
%	SECTION TITLE
%-------------------------------------------------------------------------------
\cvsection{Projects}


%-------------------------------------------------------------------------------
%	CONTENT
%-------------------------------------------------------------------------------
\begin{cventries}

%---------------------------------------------------------
  \cventry
    {Sound and Image Processing using Deep Learning} % Role
    {Speech2Face} % Event
    {\href{https://github.com/ravising-h/Speech2Face}{Link}} % Location
    {Implementin Research Paper.} % Date(s)
    {
      \begin{cvitems} % Description(s)
        \item Designed a neural network model that takes the complex spectrogram of a short speech segment as input and predicts a feature vector representing the face. More specifically, the predicted face feature represents a 4096-D face feature that is extracted from the penultimate layer (i.e., one layer prior to the classification layer) of a pre-trained face recognition.
        \item  This work is to study to what extent we can infer how a person looks from the way they talk. Specifically, from a short input audio segment of a person speaking, our method directly reconstructs an image of the person’s face in a canonical form.
       \end{cvitems}
    }


  \cventry
    {Sound Processing using Deep Learning} % Role
    {UrbanSound8K} % Event
    {\href{https://github.com/ravising-h/Urbansound8k}{Link}} % Location
    {} % Date(s)
    {
      \begin{cvitems} % Description(s)
        \item Urban Sound Classification project was performed on a data set called urbansound8K which has 10 classes of sound such as car, AC etc. For classification I have used ANN and XGBOOST, random forest and stacked model.
        \item  Concepts involved were nyquist sampling theorem, ANN, Mel-frequency cepstral coefficients.
      \end{cvitems}
    }
  \cventry
    {Image Caption} % Role
    {Assist Me} % Event
    {\href{https://github.com/ravising-h/AssistME}{Link}} % Location
    {} % Date(s)
    {
      \begin{cvitems} % Description(s)
       \item I have BUILD a Deep Learning Model which can describe an image based on a research paper by Fee Fee li.
        \item In this project there are two major parts Object recognition, Caption generation. It is a multi-modal project. I used ResNet50 for object  recognition.  The output  of ResNet50 and language model is merge and feed-ed to a LSTM for caption generation.
        \item  Made a website where someone can upload a photo and see its caption. Generate a voice which can read out the caption (Amazon Poly).
      \end{cvitems}
    }
    

  \cventry
    {Image Processing using Deep Learning} % Role
    {Neural Style Transfer} % Event
    {\href{https://github.com/ravising-h/Neural-Style-Transfer}{Link}} % Location
    {} % Date(s)
    {
      \begin{cvitems} % Description(s)
        \item Neural style transfer is an optimization technique used to take two images—a content image and a style reference image (such as an artwork by a famous painter)—and blend them together so the output image looks like the content image, but “painted” in the style of the style reference image.
        \item  Libraries used were  TensorFlow, Keras,  opencv, PIL
        \item  This is implemented by optimizing the output image to match the content statistics of the content image and the style statistics of the style reference image. These statistics are extracted from the images using a convolutional network.
      \end{cvitems}
    }
 
  \cventry
    %{OCR using Deep Learning} % Role
    {Reading RC and Speedometer and detecting Damage of a car}
    {Car Assesement}% Event
    {\href{https://github.com/ravising-h/Car-Damage-Detection-Using-Detectron}{Link}} % Location
    {} % Date(s)
    {
      \begin{cvitems} % Description(s)
        \item Detected Car Damage using  Facebook's detectron.
        \item Used YOLOv3 to detect the reading of Odometer.
        \item  Used Google's API to read RC of a car and then Regex to take out useful information from it.
      \end{cvitems}
    }
%---------------------------------------------------------
% \cventry
%     {Personal} % Role
%     {Alchehistory ({\tiny play.google.com/store/apps/details?id=com.SmoketreeStudios.Trees.Alchehistory})} % Event
%     {} % Location
%     {} % Date(s)
%     {
%       A simple Match 3 game revolving around combining elements to form other elements and vice versa. This game was developed in unity3d and featured a selection of small self contained levels and an infinite mode of play.
%     }



\end{cventries}


